\chapter{Motivation for Numerical Methods}
Many partial differential equations do not have exact closed-form solutions for all choices of initial conditions\footnote{An example is the Navier-Stokes equation which is thought to describe the motion of an incompressible viscous fluid.}. Irregular boundary conditions can also make finding an analytic solution difficult for many partial differentail equation. In these cases, finding an approximate solution with a numerical method can be helpful either for physical purposes, engineering purposes or for mathematical investigations of the behavior of solutions to these partial differential equations.
There are also cases where the partial differential equations have explicitly known exact solutions, but the formulae used to express the exact solutions require a large number of computations to evaluate them\footnote{An example is the sine-Gordon equation.}. In this case we are interested in making numerical approximations that result in accurate and cost-efficient solutions.

Numerical methods allows us to use a computer to calculate approximate solutions to partial differential equations. The accuracy of the solution will depend on which numerical method is used and usually more accurate numerical methods tend to be more complicated than less accurate methods. We will therefore start with some simple numerical methods to familiarize ourselves with how numerical methods work.
We encourage the reader to take a full course on the numerical solution of partial differential equations as well as reading the references to find out about numerical techniques not discussed here.

